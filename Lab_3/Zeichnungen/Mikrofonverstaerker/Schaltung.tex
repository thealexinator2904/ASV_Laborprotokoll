\begin{figure}[H]
    \centering
    \begin{circuitikz}[]
        \draw (0,0) node[op amp,yscale=-1] (opamp) {\scalebox{1}[-1]{$LM358$}};
        \draw (opamp.down) --++(0,0.5) node[vcc]{$V_{CC}$};
        \draw (opamp.up) --++(0,-0.5) node[vee]{$V_{EE}$};

        \draw (opamp.+) to[short] ++(-2,0) to[C=$C_1$,-o] ++(-2,0) node[left]{$U_{in}$};
        \draw (-3,0.5) to[R=$R_5$,*-] ++(0,2) node[vcc]{$V_{CC}$};
        \draw (-3,0.5) to[R=$R_4$,*-] ++(0,-2) node[ground]{};
        
        \draw (opamp.-) to[short] ++(-0.5,0)
            to[short] ++(0,-3)
            to[R=$R_2$] ++(0,-1.5)
            to[C=$C_3$] ++(0,-1.5) node[ground]{};
        \draw (opamp.out) to[short] ++(2,0)
            to[C=$C_2$,-o] ++(1.5,0) node[right]{$U_{out}$};
        \draw (3,0) to[R=$R_3$,*-] ++(0,-2) node[ground]{};
        \draw (2,0) to[short, *-] ++(0,-2.5)
            to[R=$R_1$,-*] ++(-3.7,0);
        \end{circuitikz}
    \caption{Mikrofonverstärker mit LM358 und AD823}
    \label{fig:Mikrofon_Schaltung}
 \end{figure}
