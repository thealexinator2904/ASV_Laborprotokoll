\begin{figure}[H]
    \centering
    \begin{circuitikz}[]
        %%Verstärker und Versorgung
        \draw (0,0) node[op amp] (opamp) {$\mu A 741$};
        \draw (opamp.up) --++(0,0.5) node[vcc]{$V_{CC}$};
        \draw (opamp.down) --++(0,-0.5) node[vee]{$V_{EE}$};
        
        \draw (opamp.+) to[short] ++(-0.25,0)
            to[short] ++(0,-0.25) node[ground] {};
        
        \draw (opamp.-) to[short] (-2,0.5)
            to[R=$R_1$,*-o] ++(-2,0)
            %%Einfügen der Messschaltung
            to[sV=CH1, color=white, name=S1,o-o] ++(0,-2) node[ground] {}
            to[short,o-] ++(-2,0)
            to[sV] ++(0,2)
            to[short,-o] ++(2,0);        
        \draw (-2,0.5) to[short] ++(0,2)
            to[R=$R_2$] ++(4,0)
            to[short,-*] ++(0,-2.5);
        \draw (opamp.out) to[short,-o] ++(2,0)        
            %Einfügen der Messchaltung
            to[sV=CH2, color=white, name=S2,o-o] ++(0,-2) node[ground]{};
        \myscope{S1}{0}
        \myscope{S2}{0}
        \end{circuitikz}
    \caption{Invertierender Verstärker, Messaufbau}
    \label{fig:inv_verst_Schaltung_Messaufbau}
 \end{figure}
