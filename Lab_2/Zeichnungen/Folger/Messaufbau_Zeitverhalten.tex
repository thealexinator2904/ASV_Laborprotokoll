\begin{figure}[H]
    \centering
    \begin{circuitikz}[]
        \draw (0,0) node[op amp] (opamp) {$LM358$};
        \draw (opamp.up) --++(0,0.5) node[vcc]{$V_{CC}$};
        \draw (opamp.down) --++(0,-0.5) node[vee]{$V_{EE}$};
        \draw (opamp.+) to[short,-o] ++(-1,0) 
            %%Einfügen der Messschaltung
            to[sV=CH1, color=white, name=S1,o-o] ++(0,-2) node[ground] {}
            to[short,o-] ++(-2,0)
            to[sV] ++(0,2)
            to[short,-o] ++(2,0);
        \draw (opamp.out) to[short,-o] ++(2,0)
            %Einfügen der Messchaltung
            to[sV=CH2, color=white, name=S2,o-o] ++(0,-2) node[ground]{};
        \draw (opamp.-) to[short] ++(-1,0)
            to[short] ++(0,2)
            to[short] ++(4,0)
            to[short,-*] ++(0,-2.5);
        \myscope{S1}{0}
        \myscope{S2}{0}
        \end{circuitikz}
    \caption{Spannungsfolger mit LM358, Messaufbau zur Bestimmung des zeitlichen Verhaltens}
    \label{fig:Spannungsfolger_LM358_Messaufbau_Offset}
 \end{figure}
