\begin{figure}[H]
    \centering
    \begin{circuitikz}[]
        \draw (0,0) node[op amp] (opamp) {$LM358$};
        \draw (opamp.up) --++(0,0.5) node[vcc]{$V_{CC}$};
        \draw (opamp.down) --++(0,-0.5) node[vee]{$V_{EE}$};
        \draw (opamp.+) to[short] ++(-0.5, 0) to[short] ++(0, -0.5) node[ground]{}; 
        
        \draw (opamp.-) to[R=$R_1$,-o] ++(-2,0) node[left]{$U_{in}$};
        \draw (opamp.-) to[short, *-] ++(0,2)
            to[R=$R_2$,*-*] ++(3,0)
            to[short,-*] ++(0,-2.5)
            to[short] (opamp.out);
        \draw (-1.2,2) to[R=$R_{Ref}$,*-o] ++(-2,0) node[left]{$U_{ref}$};
        \draw (opamp.-) to[short, *-] ++(0,3.5)
            to[C=$C$] ++(3,0)
            to[short] ++(0,-2);
        \draw (opamp.out) to[short, -o] (3,0) node[right]{$U_{out}$};
        \end{circuitikz}
    \caption{mögliche Schaltung zum Offsetabgleich}
    \label{fig:offsetabgleich_Tiefpass}
 \end{figure}
