\begin{figure}[H]
    \centering
    \begin{circuitikz}[]
        \draw (0,0) node[op amp] (opamp) {$LM358$};
        \draw (opamp.up) --++(0,0.5) node[vcc]{$V_{CC}$};
        \draw (opamp.down) --++(0,-0.5) node[vee]{$V_{EE}$};
        \draw (opamp.+) to[short] ++(-0.5, 0) to[short] ++(0, -0.5) node[ground]{}; 
        
        \draw (opamp.-) to[R=$R_1$] ++(-3,0)
            to[sV=CH1, color=white, name=S1,o-o] ++(0,-2) node[ground] {}
            to[short,o-] ++(-2,0)
            to[sV] ++(0,2)
            to[short,-o] ++(2,0);
        \draw (opamp.-) to[short, *-] ++(0,2)
            to[R=$R_2$,*-*] ++(3,0)
            to[short,-*] ++(0,-2.5)
            to[short] (opamp.out);
        \draw (opamp.-) to[short, *-] ++(0,3.5)
            to[C=$C$] ++(3,0)
            to[short] ++(0,-2);
        \draw (opamp.out) to[short, -o] (3,0) 
            to[sV=CH2, color=white, name=S2,o-o] ++(0,-2) node[ground]{};
        
        \myscope{S1}{0}
        \myscope{S2}{0}
        \end{circuitikz}
    \caption{Messaufbau, aktiver Tiefpass erster Ordnung}
    \label{fig:Spannungsfolger_LM358_Schaltung}
 \end{figure}
